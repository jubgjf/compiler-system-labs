\documentclass{article}

\usepackage[heading=true]{ctex}
\usepackage{float}
\usepackage{graphicx}
\usepackage{geometry}
\usepackage{dirtree}
\usepackage{listings}
\usepackage{xcolor}
\usepackage{qtree}

\ctexset{
    section={
        number=\chinese{section},
        format+=\raggedright
    }
}
\geometry{a4paper, scale=0.7}

\definecolor{codegreen}{rgb}{0,0.6,0}
\definecolor{codegray}{rgb}{0.5,0.5,0.5}
\definecolor{codepurple}{rgb}{0.58,0,0.82}
\definecolor{backcolour}{rgb}{0.95,0.95,0.92}

\lstdefinestyle{lststyle}{
    backgroundcolor=\color{backcolour},   
    commentstyle=\color{codegreen},
    keywordstyle=\color{magenta},
    numberstyle=\tiny\color{codegray},
    stringstyle=\color{codepurple},
    basicstyle=\ttfamily\footnotesize,
    breakatwhitespace=false,
    breaklines=true,
    captionpos=b,
    keepspaces=true,
    numbers=left,
    numbersep=5pt,
    showspaces=false,
    showstringspaces=false,
    showtabs=false,
    tabsize=2
}

\lstset{style=lststyle}


\title{实验三: 中间代码生成}
\author{1190200703 管健男}
\date{}


\begin{document}

\maketitle


\section{实现功能}

能够生成中间代码,可以选择打印到文件还是stdout。

\section{编译方式}

源代码提供了Makefile帮助编译,目录结构为

\begin{figure}[H]
    \centering
    \begin{minipage}{0.4\linewidth}
        \dirtree{%
            .1 compiler-system-labs.
            .2 lab3.
            .3 inter.c.
            .3 inter.h.
            .3 lexical.l.
            .3 main.c.
            .3 Makefile.
            .3 README.md.
            .3 semantic.c.
            .3 semantic.h.
            .3 syntax.y.
            .3 tree.c.
            .3 tree.h.
            .2 Makefile.
            .2 README.md.
            .2 report.
            .3 lab3.
            .4 main.pdf.
            .2 test.
            .3 \ldots.
        }
    \end{minipage}
\end{figure}

其中最外层目录compiler-system-labs/下的Makefile将调用lab3/目录下的Makefile,
因此只需在compiler-system-labs/下执行make run3即可运行实验三的中间代码生成程序。

\section{亮点内容}

采用了双向链表式的IR存储。

\begin{lstlisting}[language=C, caption=双向链表节点结构体声明]
    /// 所有中间代码组成的双向链表的一个节点
    struct inter_codes {
        struct inter_code*   code;
        struct inter_codes*  prev;
        struct inter_codes*  next;
    };
\end{lstlisting}

\begin{lstlisting}[language=C, caption=双向链表节点结构体声明]
    /// 所有中间代码组成的双向链表
    struct inter_code_list {
        struct inter_codes* head;
        struct inter_codes* tail;
    };
\end{lstlisting}

优化部分,对中间代码的生成没有做优化。因此可能有一些临时变量冗余。

\end{document}
