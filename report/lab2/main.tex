\documentclass{article}

\usepackage[heading=true]{ctex}
\usepackage{float}
\usepackage{graphicx}
\usepackage{geometry}
\usepackage{dirtree}
\usepackage{listings}
\usepackage{xcolor}
\usepackage{qtree}

\ctexset{
    section={
        number=\chinese{section},
        format+=\raggedright
    }
}
\geometry{a4paper, scale=0.7}

\definecolor{codegreen}{rgb}{0,0.6,0}
\definecolor{codegray}{rgb}{0.5,0.5,0.5}
\definecolor{codepurple}{rgb}{0.58,0,0.82}
\definecolor{backcolour}{rgb}{0.95,0.95,0.92}

\lstdefinestyle{lststyle}{
    backgroundcolor=\color{backcolour},   
    commentstyle=\color{codegreen},
    keywordstyle=\color{magenta},
    numberstyle=\tiny\color{codegray},
    stringstyle=\color{codepurple},
    basicstyle=\ttfamily\footnotesize,
    breakatwhitespace=false,
    breaklines=true,
    captionpos=b,
    keepspaces=true,
    numbers=left,
    numbersep=5pt,
    showspaces=false,
    showstringspaces=false,
    showtabs=false,
    tabsize=2
}

\lstset{style=lststyle}


\title{实验二: 语义分析}
\author{1190200703 管健男}
\date{}


\begin{document}

\maketitle


\section{实现功能}

能够检查实验指导书上的所有17种错误类型。当出现语义分析错误时,将输出
"Error type 错误码 at Line 行号: 错误提示."

\section{编译方式}

源代码提供了Makefile帮助编译,目录结构为

\begin{figure}[H]
    \centering
    \begin{minipage}{0.4\linewidth}
        \dirtree{%
            .1 compiler-system-labs.
            .2 lab2.
            .3 lexical.l.
            .3 main.c.
            .3 Makefile.
            .3 README.md.
            .3 semantic.c.
            .3 semantic.h.
            .3 syntax.y.
            .3 tree.c.
            .3 tree.h.
            .2 Makefile.
            .2 README.md.
            .2 report.
            .3 lab2.
            .4 main.pdf.
            .2 test.
            .3 \ldots.
        }
    \end{minipage}
\end{figure}

其中最外层目录compiler-system-labs/下的Makefile将调用lab2/目录下的Makefile,
因此只需在compiler-system-labs/下执行make run2即可运行实验二的语法分析程序。

\section{亮点内容}

\subsection{类型检查}

代表类型的结构体声明如下。

\begin{lstlisting}[language=C, caption=类型结构体声明]
/// 类型
struct type {
    // 标识符类型
    enum kind kind;

    // 具体的类型由 kind 字段指示
    union {
        // 基本类型
        enum basic_type basic;

        // 数组类型
        struct {
            struct type* element_type; // 元素类型
            int          size;         // 数组大小
        } array;

        // 结构体类型
        struct {
            char*         name;  // 结构体名
            struct field* field; // 结构体各个字段链表
        } structure;

        // 函数类型
        struct {
            int           argc;        // 参数数量
            struct field* argv;        // 参数链表
            struct type*  return_type; // 返回类型
        } function;
    } u;
};
\end{lstlisting}

其中采用 union 同时代表基本类型、数组类型、结构体类型、函数类型,
并由 kind 字段标识其具体类型。

其中代表数组类型的字段也是一个结构体,分别代表这个数组的元素类型,
以及这个数组当前维度的大小。
元素类型是一个继承属性,需要语法分析树的左兄弟节点提供其类型。

代表结构体类型的字段是 structure 结构体,包含两个成员变量。
name 字段代表这个结构体的名称,field 字段是结构体各个字段组成的链表的头指针。
field 字段声明如下:

\begin{lstlisting}[language=C, caption=field 结构体声明]
/// 域:带有自己的名字和类型的“类型”
struct field {
    char*         name; // 名字
    struct type*  type; // 类型
    struct field* next; // 下一个域
};
\end{lstlisting}

field代表一个有自己名字和类型的“类型”,称为“域”。
可以将多个不同类型、不同名称的变量组合成一个链表,例如结构体各个字段和函数的各个参数。

代表函数类型的字段是 function 结构体,包含三个成员变量。
argc代表参数数量。argv是一个field类型的链表,各个参数依次存放至此。
return\_type代表这个函数的返回类型,可以据此检查函数的返回值是否与接收返回值的变量匹配。

\subsection{符号表}

符号表本质上是一个哈希表。对变量名字符串进行哈希值计算,得到其在哈希表中的索引下标。
哈希表中的每一个条目是一个链表,链表中的值是域信息。哈希表(符号表)的声明如下。

\begin{lstlisting}[language=C, caption=符号表声明]
/// 符号表,本质上是一个哈希表,用于根据符号名称快速找到对应的符号表条目
struct sym_table {
    /// 一个符号表条目 item 在表中的索引位置的计算方法是:
    /// hash_array[HASH(`name`)] -> `name` 对应的符号表条目
    struct table_item** hash_array;
};
\end{lstlisting}

\end{document}
