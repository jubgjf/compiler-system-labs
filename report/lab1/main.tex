\documentclass{article}

\usepackage[heading=true]{ctex}
\usepackage{float}
\usepackage{graphicx}
\usepackage{geometry}
\usepackage{dirtree}
\usepackage{listings}
\usepackage{xcolor}
\usepackage{qtree}

\ctexset{
    section={
        number=\chinese{section},
        format+=\raggedright
    }
}
\geometry{a4paper, scale=0.7}

\definecolor{codegreen}{rgb}{0,0.6,0}
\definecolor{codegray}{rgb}{0.5,0.5,0.5}
\definecolor{codepurple}{rgb}{0.58,0,0.82}
\definecolor{backcolour}{rgb}{0.95,0.95,0.92}

\lstdefinestyle{lststyle}{
    backgroundcolor=\color{backcolour},   
    commentstyle=\color{codegreen},
    keywordstyle=\color{magenta},
    numberstyle=\tiny\color{codegray},
    stringstyle=\color{codepurple},
    basicstyle=\ttfamily\footnotesize,
    breakatwhitespace=false,
    breaklines=true,
    captionpos=b,
    keepspaces=true,
    numbers=left,
    numbersep=5pt,
    showspaces=false,
    showstringspaces=false,
    showtabs=false,
    tabsize=2
}

\lstset{style=lststyle}


\title{实验一: 词法分析与语法分析}
\author{1190200703 管健男}
\date{}


\begin{document}

\maketitle


\section{实现功能}

\subsection{词法分析}

可以识别的Token包括
INT, FLOAT, STRUCT, RETURN, IF, ELSE, WHILE, TYPE, ID, SEMI, COMMA, ASSIGNOP,
RELOP, PLUS, MINUS, STAR, DIV, AND, OR, DOT, NOT, LP, RP, LB, RB, LC, RC,
其中INT为十进制数,FLOAT支持含有"e"的指数形式。

可以识别"//"格式的注释。

当出现词法分析错误时,将输出"Error type A at Line 行号."

\subsection{语法分析}

可以按照实验指导书规定的文法打印语法树。

当出现词法分析错误时,将输出"Error type B at Line 行号."

\section{编译方式}

源代码提供了Makefile帮助编译,目录结构为

\begin{figure}[H]
    \centering
    \begin{minipage}{0.4\linewidth}
        \dirtree{%
            .1 compiler-system-labs.
            .2 lab1.
            .3 lexical.l.
            .3 main.c.
            .3 Makefile.
            .3 README.md.
            .3 syntax.y.
            .3 tree.c.
            .3 tree.h.
            .2 Makefile.
            .2 README.md.
            .2 report.
            .3 lab1.
            .4 main.pdf.
            .2 test.
            .3 test-lexical.txt.
            .3 test-syntax.txt.
        }
    \end{minipage}
\end{figure}

其中最外层目录compiler-system-labs/下的Makefile将调用lab1/目录下的Makefile,
因此只需在compiler-system-labs/下执行make run1即可运行实验一的语法分析程序。

\section{亮点内容}

\subsection{语法树的数据结构}

语法树不是一颗二叉树,一个节点可能有多个子节点,如下图所示。

\Tree [.Exp EXP ASSIGNOP [.Exp [.Exp id ] [.PLUS ] [.Exp INT ] ].Exp ].Exp

可以采用两种方案:第一种是在一个节点中,存放其最左孩子和其右兄弟的指针。
第二种是将这棵树转化为二叉树,再利用转换前后前序遍历输出的结果相同这一特性,打印出语法树。
在本实验中,采用第一种方案,代码如下

\begin{lstlisting}[language=C, caption=语法树节点数据结构]
struct node {
    int          lineno;        // 节点对应 token 所在的行号
    enum token   token;         // token 类别
    char         name[32];      // token 展示的名称
    struct node* left_child;    // 节点的最左子节点
    struct node* right_sibling; // 节点的右兄弟节点
};
\end{lstlisting}

这样,当需要找到一个节点的所有子节点时,只需遍历其最左孩子的所有右兄弟节点即可。

\subsection{可变参数}

由于每一个节点的子节点数量都可能不一致,因此在创建节点时,传入的子节点数量可变。

使用了C语言的va\_list(定义在stdarg.h)从而支持此功能。

\end{document}
